
% Default to the notebook output style




% Inherit from the specified cell style.





\documentclass[11pt]{article}

    \usepackage[top=1cm, bottom=2cm, left=0cm, right=0cm]{geometry}

    \usepackage{multicol}
    \setlength{\columnseprule}{0.1mm}    % for two columns
    \setlength{\columnsep}{.6cm}         % for two columns

    \usepackage{kotex}
    \usepackage[T1]{fontenc}
    % Nicer default font (+ math font) than Computer Modern for most use cases
    \usepackage{mathpazo}

    % Basic figure setup, for now with no caption control since it's done
    % automatically by Pandoc (which extracts ![](path) syntax from Markdown).
    \usepackage{graphicx}
    % We will generate all images so they have a width \maxwidth. This means
    % that they will get their normal width if they fit onto the page, but
    % are scaled down if they would overflow the margins.
    \makeatletter
    \def\maxwidth{\ifdim\Gin@nat@width>\linewidth\linewidth
    \else\Gin@nat@width\fi}
    \makeatother
    \let\Oldincludegraphics\includegraphics
    % Set max figure width to be 80% of text width, for now hardcoded.
    \renewcommand{\includegraphics}[1]{\Oldincludegraphics[width=.8\maxwidth]{#1}}
    % Ensure that by default, figures have no caption (until we provide a
    % proper Figure object with a Caption API and a way to capture that
    % in the conversion process - todo).
    \usepackage{caption}
    \DeclareCaptionLabelFormat{nolabel}{}
    \captionsetup{labelformat=nolabel}

    \usepackage{adjustbox} % Used to constrain images to a maximum size
    \usepackage{xcolor} % Allow colors to be defined
    \usepackage{enumerate} % Needed for markdown enumerations to work
    \usepackage{geometry} % Used to adjust the document margins
    \usepackage{amsmath} % Equations
    \usepackage{amssymb} % Equations
    \usepackage{textcomp} % defines textquotesingle
    % Hack from http://tex.stackexchange.com/a/47451/13684:
    \AtBeginDocument{%
        \def\PYZsq{\textquotesingle}% Upright quotes in Pygmentized code
    }
    \usepackage{upquote} % Upright quotes for verbatim code
    \usepackage{eurosym} % defines \euro
    \usepackage[mathletters]{ucs} % Extended unicode (utf-8) support
%    \usepackage[utf8x]{inputenc} % Allow utf-8 characters in the tex document
%    \usepackage{inputenc} % Allow utf-8 characters in the tex document
    \usepackage{fancyvrb} % verbatim replacement that allows latex
    \usepackage{grffile} % extends the file name processing of package graphics
                         % to support a larger range
    % The hyperref package gives us a pdf with properly built
    % internal navigation ('pdf bookmarks' for the table of contents,
    % internal cross-reference links, web links for URLs, etc.)
    \usepackage{hyperref}
    \usepackage{longtable} % longtable support required by pandoc >1.10
    \usepackage{booktabs}  % table support for pandoc > 1.12.2
    \usepackage[inline]{enumitem} % IRkernel/repr support (it uses the enumerate* environment)
    \usepackage[normalem]{ulem} % ulem is needed to support strikethroughs (\sout)
                                % normalem makes italics be italics, not underlines




    % Colors for the hyperref package
    \definecolor{urlcolor}{rgb}{0,.145,.698}
    \definecolor{linkcolor}{rgb}{.71,0.21,0.01}
    \definecolor{citecolor}{rgb}{.12,.54,.11}

    % ANSI colors
    \definecolor{ansi-black}{HTML}{3E424D}
    \definecolor{ansi-black-intense}{HTML}{282C36}
    \definecolor{ansi-red}{HTML}{E75C58}
    \definecolor{ansi-red-intense}{HTML}{B22B31}
    \definecolor{ansi-green}{HTML}{00A250}
    \definecolor{ansi-green-intense}{HTML}{007427}
    \definecolor{ansi-yellow}{HTML}{DDB62B}
    \definecolor{ansi-yellow-intense}{HTML}{B27D12}
    \definecolor{ansi-blue}{HTML}{208FFB}
    \definecolor{ansi-blue-intense}{HTML}{0065CA}
    \definecolor{ansi-magenta}{HTML}{D160C4}
    \definecolor{ansi-magenta-intense}{HTML}{A03196}
    \definecolor{ansi-cyan}{HTML}{60C6C8}
    \definecolor{ansi-cyan-intense}{HTML}{258F8F}
    \definecolor{ansi-white}{HTML}{C5C1B4}
    \definecolor{ansi-white-intense}{HTML}{A1A6B2}

    % commands and environments needed by pandoc snippets
    % extracted from the output of `pandoc -s`
    \providecommand{\tightlist}{%
      \setlength{\itemsep}{0pt}\setlength{\parskip}{0pt}}
    \DefineVerbatimEnvironment{Highlighting}{Verbatim}{commandchars=\\\{\}}
    % Add ',fontsize=\small' for more characters per line
    \newenvironment{Shaded}{}{}
    \newcommand{\KeywordTok}[1]{\textcolor[rgb]{0.00,0.44,0.13}{\textbf{{#1}}}}
    \newcommand{\DataTypeTok}[1]{\textcolor[rgb]{0.56,0.13,0.00}{{#1}}}
    \newcommand{\DecValTok}[1]{\textcolor[rgb]{0.25,0.63,0.44}{{#1}}}
    \newcommand{\BaseNTok}[1]{\textcolor[rgb]{0.25,0.63,0.44}{{#1}}}
    \newcommand{\FloatTok}[1]{\textcolor[rgb]{0.25,0.63,0.44}{{#1}}}
    \newcommand{\CharTok}[1]{\textcolor[rgb]{0.25,0.44,0.63}{{#1}}}
    \newcommand{\StringTok}[1]{\textcolor[rgb]{0.25,0.44,0.63}{{#1}}}
    \newcommand{\CommentTok}[1]{\textcolor[rgb]{0.38,0.63,0.69}{\textit{{#1}}}}
    \newcommand{\OtherTok}[1]{\textcolor[rgb]{0.00,0.44,0.13}{{#1}}}
    \newcommand{\AlertTok}[1]{\textcolor[rgb]{1.00,0.00,0.00}{\textbf{{#1}}}}
    \newcommand{\FunctionTok}[1]{\textcolor[rgb]{0.02,0.16,0.49}{{#1}}}
    \newcommand{\RegionMarkerTok}[1]{{#1}}
    \newcommand{\ErrorTok}[1]{\textcolor[rgb]{1.00,0.00,0.00}{\textbf{{#1}}}}
    \newcommand{\NormalTok}[1]{{#1}}

    % Additional commands for more recent versions of Pandoc
    \newcommand{\ConstantTok}[1]{\textcolor[rgb]{0.53,0.00,0.00}{{#1}}}
    \newcommand{\SpecialCharTok}[1]{\textcolor[rgb]{0.25,0.44,0.63}{{#1}}}
    \newcommand{\VerbatimStringTok}[1]{\textcolor[rgb]{0.25,0.44,0.63}{{#1}}}
    \newcommand{\SpecialStringTok}[1]{\textcolor[rgb]{0.73,0.40,0.53}{{#1}}}
    \newcommand{\ImportTok}[1]{{#1}}
    \newcommand{\DocumentationTok}[1]{\textcolor[rgb]{0.73,0.13,0.13}{\textit{{#1}}}}
    \newcommand{\AnnotationTok}[1]{\textcolor[rgb]{0.38,0.63,0.69}{\textbf{\textit{{#1}}}}}
    \newcommand{\CommentVarTok}[1]{\textcolor[rgb]{0.38,0.63,0.69}{\textbf{\textit{{#1}}}}}
    \newcommand{\VariableTok}[1]{\textcolor[rgb]{0.10,0.09,0.49}{{#1}}}
    \newcommand{\ControlFlowTok}[1]{\textcolor[rgb]{0.00,0.44,0.13}{\textbf{{#1}}}}
    \newcommand{\OperatorTok}[1]{\textcolor[rgb]{0.40,0.40,0.40}{{#1}}}
    \newcommand{\BuiltInTok}[1]{{#1}}
    \newcommand{\ExtensionTok}[1]{{#1}}
    \newcommand{\PreprocessorTok}[1]{\textcolor[rgb]{0.74,0.48,0.00}{{#1}}}
    \newcommand{\AttributeTok}[1]{\textcolor[rgb]{0.49,0.56,0.16}{{#1}}}
    \newcommand{\InformationTok}[1]{\textcolor[rgb]{0.38,0.63,0.69}{\textbf{\textit{{#1}}}}}
    \newcommand{\WarningTok}[1]{\textcolor[rgb]{0.38,0.63,0.69}{\textbf{\textit{{#1}}}}}


    % Define a nice break command that doesn't care if a line doesn't already
    % exist.
    \def\br{\hspace*{\fill} \\* }
    % Math Jax compatability definitions
    \def\gt{>}
    \def\lt{<}
    % Document parameters
    \title{2017년 2학기 공업수학 중간고사}
    \author{이름: \hspace{4cm} 학번: \hspace{4cm} 점수:\hspace{2cm}/ 100점 만점 }
    \date{}
    % Pygments definitions

\makeatletter
\def\PY@reset{\let\PY@it=\relax \let\PY@bf=\relax%
    \let\PY@ul=\relax \let\PY@tc=\relax%
    \let\PY@bc=\relax \let\PY@ff=\relax}
\def\PY@tok#1{\csname PY@tok@#1\endcsname}
\def\PY@toks#1+{\ifx\relax#1\empty\else%
    \PY@tok{#1}\expandafter\PY@toks\fi}
\def\PY@do#1{\PY@bc{\PY@tc{\PY@ul{%
    \PY@it{\PY@bf{\PY@ff{#1}}}}}}}
\def\PY#1#2{\PY@reset\PY@toks#1+\relax+\PY@do{#2}}

\expandafter\def\csname PY@tok@gd\endcsname{\def\PY@tc##1{\textcolor[rgb]{0.63,0.00,0.00}{##1}}}
\expandafter\def\csname PY@tok@gu\endcsname{\let\PY@bf=\textbf\def\PY@tc##1{\textcolor[rgb]{0.50,0.00,0.50}{##1}}}
\expandafter\def\csname PY@tok@gt\endcsname{\def\PY@tc##1{\textcolor[rgb]{0.00,0.27,0.87}{##1}}}
\expandafter\def\csname PY@tok@gs\endcsname{\let\PY@bf=\textbf}
\expandafter\def\csname PY@tok@gr\endcsname{\def\PY@tc##1{\textcolor[rgb]{1.00,0.00,0.00}{##1}}}
\expandafter\def\csname PY@tok@cm\endcsname{\let\PY@it=\textit\def\PY@tc##1{\textcolor[rgb]{0.25,0.50,0.50}{##1}}}
\expandafter\def\csname PY@tok@vg\endcsname{\def\PY@tc##1{\textcolor[rgb]{0.10,0.09,0.49}{##1}}}
\expandafter\def\csname PY@tok@vi\endcsname{\def\PY@tc##1{\textcolor[rgb]{0.10,0.09,0.49}{##1}}}
\expandafter\def\csname PY@tok@vm\endcsname{\def\PY@tc##1{\textcolor[rgb]{0.10,0.09,0.49}{##1}}}
\expandafter\def\csname PY@tok@mh\endcsname{\def\PY@tc##1{\textcolor[rgb]{0.40,0.40,0.40}{##1}}}
\expandafter\def\csname PY@tok@cs\endcsname{\let\PY@it=\textit\def\PY@tc##1{\textcolor[rgb]{0.25,0.50,0.50}{##1}}}
\expandafter\def\csname PY@tok@ge\endcsname{\let\PY@it=\textit}
\expandafter\def\csname PY@tok@vc\endcsname{\def\PY@tc##1{\textcolor[rgb]{0.10,0.09,0.49}{##1}}}
\expandafter\def\csname PY@tok@il\endcsname{\def\PY@tc##1{\textcolor[rgb]{0.40,0.40,0.40}{##1}}}
\expandafter\def\csname PY@tok@go\endcsname{\def\PY@tc##1{\textcolor[rgb]{0.53,0.53,0.53}{##1}}}
\expandafter\def\csname PY@tok@cp\endcsname{\def\PY@tc##1{\textcolor[rgb]{0.74,0.48,0.00}{##1}}}
\expandafter\def\csname PY@tok@gi\endcsname{\def\PY@tc##1{\textcolor[rgb]{0.00,0.63,0.00}{##1}}}
\expandafter\def\csname PY@tok@gh\endcsname{\let\PY@bf=\textbf\def\PY@tc##1{\textcolor[rgb]{0.00,0.00,0.50}{##1}}}
\expandafter\def\csname PY@tok@ni\endcsname{\let\PY@bf=\textbf\def\PY@tc##1{\textcolor[rgb]{0.60,0.60,0.60}{##1}}}
\expandafter\def\csname PY@tok@nl\endcsname{\def\PY@tc##1{\textcolor[rgb]{0.63,0.63,0.00}{##1}}}
\expandafter\def\csname PY@tok@nn\endcsname{\let\PY@bf=\textbf\def\PY@tc##1{\textcolor[rgb]{0.00,0.00,1.00}{##1}}}
\expandafter\def\csname PY@tok@no\endcsname{\def\PY@tc##1{\textcolor[rgb]{0.53,0.00,0.00}{##1}}}
\expandafter\def\csname PY@tok@na\endcsname{\def\PY@tc##1{\textcolor[rgb]{0.49,0.56,0.16}{##1}}}
\expandafter\def\csname PY@tok@nb\endcsname{\def\PY@tc##1{\textcolor[rgb]{0.00,0.50,0.00}{##1}}}
\expandafter\def\csname PY@tok@nc\endcsname{\let\PY@bf=\textbf\def\PY@tc##1{\textcolor[rgb]{0.00,0.00,1.00}{##1}}}
\expandafter\def\csname PY@tok@nd\endcsname{\def\PY@tc##1{\textcolor[rgb]{0.67,0.13,1.00}{##1}}}
\expandafter\def\csname PY@tok@ne\endcsname{\let\PY@bf=\textbf\def\PY@tc##1{\textcolor[rgb]{0.82,0.25,0.23}{##1}}}
\expandafter\def\csname PY@tok@nf\endcsname{\def\PY@tc##1{\textcolor[rgb]{0.00,0.00,1.00}{##1}}}
\expandafter\def\csname PY@tok@si\endcsname{\let\PY@bf=\textbf\def\PY@tc##1{\textcolor[rgb]{0.73,0.40,0.53}{##1}}}
\expandafter\def\csname PY@tok@s2\endcsname{\def\PY@tc##1{\textcolor[rgb]{0.73,0.13,0.13}{##1}}}
\expandafter\def\csname PY@tok@nt\endcsname{\let\PY@bf=\textbf\def\PY@tc##1{\textcolor[rgb]{0.00,0.50,0.00}{##1}}}
\expandafter\def\csname PY@tok@nv\endcsname{\def\PY@tc##1{\textcolor[rgb]{0.10,0.09,0.49}{##1}}}
\expandafter\def\csname PY@tok@s1\endcsname{\def\PY@tc##1{\textcolor[rgb]{0.73,0.13,0.13}{##1}}}
\expandafter\def\csname PY@tok@dl\endcsname{\def\PY@tc##1{\textcolor[rgb]{0.73,0.13,0.13}{##1}}}
\expandafter\def\csname PY@tok@ch\endcsname{\let\PY@it=\textit\def\PY@tc##1{\textcolor[rgb]{0.25,0.50,0.50}{##1}}}
\expandafter\def\csname PY@tok@m\endcsname{\def\PY@tc##1{\textcolor[rgb]{0.40,0.40,0.40}{##1}}}
\expandafter\def\csname PY@tok@gp\endcsname{\let\PY@bf=\textbf\def\PY@tc##1{\textcolor[rgb]{0.00,0.00,0.50}{##1}}}
\expandafter\def\csname PY@tok@sh\endcsname{\def\PY@tc##1{\textcolor[rgb]{0.73,0.13,0.13}{##1}}}
\expandafter\def\csname PY@tok@ow\endcsname{\let\PY@bf=\textbf\def\PY@tc##1{\textcolor[rgb]{0.67,0.13,1.00}{##1}}}
\expandafter\def\csname PY@tok@sx\endcsname{\def\PY@tc##1{\textcolor[rgb]{0.00,0.50,0.00}{##1}}}
\expandafter\def\csname PY@tok@bp\endcsname{\def\PY@tc##1{\textcolor[rgb]{0.00,0.50,0.00}{##1}}}
\expandafter\def\csname PY@tok@c1\endcsname{\let\PY@it=\textit\def\PY@tc##1{\textcolor[rgb]{0.25,0.50,0.50}{##1}}}
\expandafter\def\csname PY@tok@fm\endcsname{\def\PY@tc##1{\textcolor[rgb]{0.00,0.00,1.00}{##1}}}
\expandafter\def\csname PY@tok@o\endcsname{\def\PY@tc##1{\textcolor[rgb]{0.40,0.40,0.40}{##1}}}
\expandafter\def\csname PY@tok@kc\endcsname{\let\PY@bf=\textbf\def\PY@tc##1{\textcolor[rgb]{0.00,0.50,0.00}{##1}}}
\expandafter\def\csname PY@tok@c\endcsname{\let\PY@it=\textit\def\PY@tc##1{\textcolor[rgb]{0.25,0.50,0.50}{##1}}}
\expandafter\def\csname PY@tok@mf\endcsname{\def\PY@tc##1{\textcolor[rgb]{0.40,0.40,0.40}{##1}}}
\expandafter\def\csname PY@tok@err\endcsname{\def\PY@bc##1{\setlength{\fboxsep}{0pt}\fcolorbox[rgb]{1.00,0.00,0.00}{1,1,1}{\strut ##1}}}
\expandafter\def\csname PY@tok@mb\endcsname{\def\PY@tc##1{\textcolor[rgb]{0.40,0.40,0.40}{##1}}}
\expandafter\def\csname PY@tok@ss\endcsname{\def\PY@tc##1{\textcolor[rgb]{0.10,0.09,0.49}{##1}}}
\expandafter\def\csname PY@tok@sr\endcsname{\def\PY@tc##1{\textcolor[rgb]{0.73,0.40,0.53}{##1}}}
\expandafter\def\csname PY@tok@mo\endcsname{\def\PY@tc##1{\textcolor[rgb]{0.40,0.40,0.40}{##1}}}
\expandafter\def\csname PY@tok@kd\endcsname{\let\PY@bf=\textbf\def\PY@tc##1{\textcolor[rgb]{0.00,0.50,0.00}{##1}}}
\expandafter\def\csname PY@tok@mi\endcsname{\def\PY@tc##1{\textcolor[rgb]{0.40,0.40,0.40}{##1}}}
\expandafter\def\csname PY@tok@kn\endcsname{\let\PY@bf=\textbf\def\PY@tc##1{\textcolor[rgb]{0.00,0.50,0.00}{##1}}}
\expandafter\def\csname PY@tok@cpf\endcsname{\let\PY@it=\textit\def\PY@tc##1{\textcolor[rgb]{0.25,0.50,0.50}{##1}}}
\expandafter\def\csname PY@tok@kr\endcsname{\let\PY@bf=\textbf\def\PY@tc##1{\textcolor[rgb]{0.00,0.50,0.00}{##1}}}
\expandafter\def\csname PY@tok@s\endcsname{\def\PY@tc##1{\textcolor[rgb]{0.73,0.13,0.13}{##1}}}
\expandafter\def\csname PY@tok@kp\endcsname{\def\PY@tc##1{\textcolor[rgb]{0.00,0.50,0.00}{##1}}}
\expandafter\def\csname PY@tok@w\endcsname{\def\PY@tc##1{\textcolor[rgb]{0.73,0.73,0.73}{##1}}}
\expandafter\def\csname PY@tok@kt\endcsname{\def\PY@tc##1{\textcolor[rgb]{0.69,0.00,0.25}{##1}}}
\expandafter\def\csname PY@tok@sc\endcsname{\def\PY@tc##1{\textcolor[rgb]{0.73,0.13,0.13}{##1}}}
\expandafter\def\csname PY@tok@sb\endcsname{\def\PY@tc##1{\textcolor[rgb]{0.73,0.13,0.13}{##1}}}
\expandafter\def\csname PY@tok@sa\endcsname{\def\PY@tc##1{\textcolor[rgb]{0.73,0.13,0.13}{##1}}}
\expandafter\def\csname PY@tok@k\endcsname{\let\PY@bf=\textbf\def\PY@tc##1{\textcolor[rgb]{0.00,0.50,0.00}{##1}}}
\expandafter\def\csname PY@tok@se\endcsname{\let\PY@bf=\textbf\def\PY@tc##1{\textcolor[rgb]{0.73,0.40,0.13}{##1}}}
\expandafter\def\csname PY@tok@sd\endcsname{\let\PY@it=\textit\def\PY@tc##1{\textcolor[rgb]{0.73,0.13,0.13}{##1}}}

\def\PYZbs{\char`\\}
\def\PYZus{\char`\_}
\def\PYZob{\char`\{}
\def\PYZcb{\char`\}}
\def\PYZca{\char`\^}
\def\PYZam{\char`\&}
\def\PYZlt{\char`\<}
\def\PYZgt{\char`\>}
\def\PYZsh{\char`\#}
\def\PYZpc{\char`\%}
\def\PYZdl{\char`\$}
\def\PYZhy{\char`\-}
\def\PYZsq{\char`\'}
\def\PYZdq{\char`\"}
\def\PYZti{\char`\~}
% for compatibility with earlier versions
\def\PYZat{@}
\def\PYZlb{[}
\def\PYZrb{]}
\makeatother


    % Exact colors from NB
    \definecolor{incolor}{rgb}{0.0, 0.0, 0.5}
    \definecolor{outcolor}{rgb}{0.545, 0.0, 0.0}




    % Prevent overflowing lines due to hard-to-break entities
    \sloppy
    % Setup hyperref package
    \hypersetup{
      breaklinks=true,  % so long urls are correctly broken across lines
      colorlinks=true,
      urlcolor=urlcolor,
      linkcolor=linkcolor,
      citecolor=citecolor,
      }
    % Slightly bigger margins than the latex defaults

    \geometry{verbose,tmargin=1in,bmargin=1in,lmargin=1in,rmargin=1in}



    \begin{document}


    \maketitle

    \begin{center}
      \begin{tabular}{|c|c|c|c|c|c|c|c|c|c|c|}
        \hline
        문제 1 & 문제 2 & 문제 3 & 문제 4 & 문제 5 & 문제 6 & 문제 7 & 문제 8 & 문제 9 & 문제 10 & 합\quad 계 \\
        \hline
        & & & & & & & & & & \\
        & & & & & & & & & & \\
        \hline
      \end{tabular}
    \end{center}

\vspace{.2cm}

\section*{시험에서 사용하는 모듈 임포트 하기}

\begin{Verbatim}[commandchars=\\\{\}]
\PY{k+kn}{import} \PY{n+nn}{matplotlib.pyplot} \PY{k+kn}{as} \PY{n+nn}{plt}
\PY{k+kn}{import} \PY{n+nn}{numpy} \PY{k+kn}{as} \PY{n+nn}{np}
\end{Verbatim}

\section*{파이썬 기본 자료형}
\label{uxd30cuxc774uxc36c-uxae30uxbcf8-uxc790uxb8ccuxd615}

\subsection*{문제 1}\label{uxbb38uxc81c}

실수(부동소수점)를 하나 입력받아, 그 숫자를 반지름으로 하는 원의 면적과
둘레의 길이를 튜플로 리턴하는 함수 \texttt{circle\_radius}를 구현하는
코드를 작성하라,

\begin{verbatim}




\end{verbatim}

\section*{문자열 자료형}\label{uxbb38uxc790uxc5f4-uxc790uxb8ccuxd615}

아래 사이트는 커피 콩의 현재 시세를 보여준다.

\begin{verbatim}
http://beans-r-us.appspot.com/prices.html
\end{verbatim}

위 사이트의 내용을 html 소스코드로 보면 다음과 같으며, 검색된 시간의
커피콩의 가격은 \texttt{Current\ price\ of\ coffee\ beans} 문장이 담겨
있는 줄에 명시되어 있다.

\begin{Shaded}
\begin{Highlighting}[]
\KeywordTok{<html><head><title>}\NormalTok{Welcome to the Beans'R'Us Pricing Page}\KeywordTok{</title>}
\KeywordTok{<link}\OtherTok{ rel=}\StringTok{"stylesheet"}\OtherTok{ type=}\StringTok{"text/css"}\OtherTok{ href=}\StringTok{"beansrus.css"} \KeywordTok{/>}
\KeywordTok{</head><body>}
\KeywordTok{<h2>}\NormalTok{Welcome to the Beans'R'Us Pricing Page}\KeywordTok{</h2>}
\KeywordTok{<p>}\NormalTok{Current price of coffee beans = }\KeywordTok{<strong>}\NormalTok{$5.94}\KeywordTok{</strong></p>}
\KeywordTok{<p>}\NormalTok{Price valid for 15 minutes from Sun Sep 10 12:21:58 2017.}\KeywordTok{</p>}
\KeywordTok{</body></html>}
\end{Highlighting}
\end{Shaded}

위 소스코드를 웹페이지의 주소만 알고 있을 때 아래와 같이 읽어들일 수 있다.

\begin{verbatim}
page = urllib2.urlopen("http://beans-r-us.appspot.com/prices.html")
text = page.read().decode("utf8")
\end{verbatim}

위 코드를 실행하면 \verb|text| 변수에 웹페이지의 전체 내용이 하나의 문자열로 저장된다.

\subsection*{문제 2}\label{uxbb38uxc81c}

\verb|text|에 저장된 문자열에서 가격 정보를 확인하는 방법을 코드를 이용하여 설명하라.


\begin{verbatim}





\end{verbatim}

\iffalse
견본답안:

\begin{Verbatim}
price_index = text.find(">$") + 2
bean_price_str = text[price_index : price_index + 4]
bean_price = float(bean_price_str)
\end{Verbatim}
\fi

\iffalse
\section*{오류 및 예외 처리}\label{uxc624uxb958-uxbc0f-uxc608uxc678-uxcc98uxb9ac}

\subsection*{문제 3}
\label{uxc544uxb798-uxcf54uxb4dcuxac00-uxd558uxb294-uxc77cuxc744-uxc124uxba85uxd558uxb77c.}

아래 코드가 하는 일을 설명하라.
\begin{verbatim}
number_to_square = raw_input("A number to divide 100: ")

try:
    number = float(number_to_square)
    print("100을 입력한 값으로 나눈 결과는", 100/number, "입니다.")
except ZeroDivisionError:
    raise ZeroDivisionError('0이 아닌 숫자를 입력하세요.')
except ValueError:
    raise ValueError('숫자를 입력하세요.')
\end{verbatim}

\begin{Verbatim}







\end{Verbatim}
\fi

\iffalse
\section*{리스트}\label{uxb9acuxc2a4uxd2b8}

\subsection*{문제 4}\label{uxbb38uxc81c}

아래 설명 중에서 리스트 자료형의 성질에 해당하는 항목을 모두 골라라.

\begin{enumerate}
\def\labelenumi{\arabic{enumi}.}
\tightlist
\item
  가변 자료형이다.
\end{enumerate}

\begin{itemize}
\tightlist
\item
  불변 자료형이다.
\item
  인덱스와 슬라이싱을 활용하여 항목의 내용을 확인하고 활용할 수 있다.
\item
  항목들이 임의의 자료형을 가질 수 있다.
\item
  리스트 길이에 제한이 있다.
\item
  신성정보 등 중요한 데이터를 보관할 때 사용한다.
\end{itemize}

\begin{verbatim}



\end{verbatim}

    견본답안: 1, 3, 4
\fi

\section*{사전}\label{uxc0acuxc804}

\texttt{record\_list.txt} 파일은 여덟 명의 수영 선수의 50m 기록을 담고
있다.

\begin{verbatim}
player1 21.09
player2 20.32
player3 21.81
player4 22.97
player5 23.29
player6 22.09
player7 21.20
player8 22.16
\end{verbatim}

\subsection*{문제 3}\label{uxbb38uxc81c}

아래코드가 하는 일을 설명하라.

\begin{Shaded}
\begin{Highlighting}[]
\NormalTok{record_f }\OperatorTok{=} \BuiltInTok{open}\NormalTok{(}\StringTok{"record_list.txt"}\NormalTok{, }\StringTok{'r'}\NormalTok{)}
\NormalTok{record }\OperatorTok{=}\NormalTok{ record_f.read().decode(}\StringTok{'utf8'}\NormalTok{).split(}\StringTok{'}\CharTok{\textbackslash{}n}\StringTok{'}\NormalTok{)}

\NormalTok{record_dict }\OperatorTok{=}\NormalTok{ \{\}}

\ControlFlowTok{for}\NormalTok{ line }\KeywordTok{in}\NormalTok{ record:}
\NormalTok{    (player, p_record) }\OperatorTok{=}\NormalTok{ line.split()}
\NormalTok{    record_dict[p_record] }\OperatorTok{=}\NormalTok{ player}

\NormalTok{record_f.close()}

\NormalTok{record_list }\OperatorTok{=}\NormalTok{ record_dict.keys()}
\NormalTok{record_list.sort()}

\end{Highlighting}
\end{Shaded}


\section*{튜플}\label{uxd29cuxd50c}

\subsection*{문제 4}\label{uxbb38uxc81c}

아래 설명 중에서 튜플 자료형의 성질에 해당하는 항목을 모두 골라라.

\begin{enumerate}
\def\labelenumi{\arabic{enumi}.}
\tightlist
\item
  가변 자료형이다.
\item
  불변 자료형이다.
\item
  인덱스와 슬라이싱을 활용하여 항목의 내용을 확인하고 활용할 수 있다.
\item
  항목들이 임의의 자료형을 가질 수 있다.
\item
  튜플 길이에 제한이 있다.
\item
  신성정보 등 중요한 데이터를 보관할 때 사용한다.
\end{enumerate}

\begin{verbatim}


\end{verbatim}

\iffalse
견본답안: 2, 3, 4, 6
\fi

\section*{리스트
조건제시법}\label{uxb9acuxc2a4uxd2b8-uxc870uxac74uxc81cuxc2dcuxbc95}

    아래 코드는 0부터 1000 사이의 홀수들의 제곱의 리스트를 조건제시법으로
생성한다

\begin{Verbatim}[commandchars=\\\{\}]
\PY{n}{odd\PYZus{}1000} \PY{o}{=} \PY{p}{[}\PY{n}{x}\PY{o}{*}\PY{o}{*}\PY{l+m+mi}{2} \PY{k}{for} \PY{n}{x} \PY{o+ow}{in} \PY{n+nb}{range}\PY{p}{(}\PY{l+m+mi}{0}\PY{p}{,} \PY{l+m+mi}{1000}\PY{p}{)} \PY{k}{if} \PY{n}{x} \PY{o}{\PYZpc{}} \PY{l+m+mi}{2} \PY{o}{==} \PY{l+m+mi}{1}\PY{p}{]}
\end{Verbatim}

\subsection*{문제 5}\label{uxbb38uxc81c}

0부터 1000까지의 숫자들 중에서 홀수이면서 7의 배수인 숫자들의 리스트를
조건제시법으로 생성하는 코드를 작성하라.

\begin{verbatim}






\end{verbatim}

\iffalse
모범답안:

\begin{Verbatim}[commandchars=\\\{\}]
{\color{incolor}In [{\color{incolor}4}]:} \PY{n}{odd\PYZus{}3x7} \PY{o}{=} \PY{p}{[}\PY{n}{x} \PY{k}{for} \PY{n}{x} \PY{o+ow}{in} \PY{n+nb}{range}\PY{p}{(}\PY{l+m+mi}{0}\PY{p}{,} \PY{l+m+mi}{1000}\PY{p}{)} \PY{k}{if} \PY{n}{x} \PY{o}{\PYZpc{}} \PY{l+m+mi}{2} \PY{o}{==} \PY{l+m+mi}{1} \PY{o+ow}{and} \PY{n}{x} \PY{o}{\PYZpc{}} \PY{l+m+mi}{7} \PY{o}{==} \PY{l+m+mi}{0}\PY{p}{]}

        \PY{c+c1}{\PYZsh{} 리스트의 처음 다섯 개 항목}
        \PY{n}{odd\PYZus{}3x7}\PY{p}{[}\PY{p}{:}\PY{l+m+mi}{5}\PY{p}{]}
\end{Verbatim}

            \begin{Verbatim}[commandchars=\\\{\}]
{\color{outcolor}Out[{\color{outcolor}4}]:} [7, 21, 35, 49, 63]
\end{Verbatim}

\fi

\subsection*{문제 6}\label{uxbb38uxc81c}

0부터 1000까지의 숫자들 중에서 홀수이면서 7의 배수인 숫자들을 제곱하여
1을 더한 값들의 리스트를 조건제시법으로 생성하는 코드를 작성하라. 힌트:
아래와 같이 정의된 함수를 활용한다.

\[f(x) = x^2 + 1\]

\begin{Verbatim}







\end{Verbatim}

\iffalse
견본답안:

\begin{Verbatim}[commandchars=\\\{\}]
{\color{incolor}In [{\color{incolor}5}]:} \PY{k}{def} \PY{n+nf}{square\PYZus{}plus1}\PY{p}{(}\PY{n}{x}\PY{p}{)}\PY{p}{:}
            \PY{k}{return} \PY{n}{x}\PY{o}{*}\PY{o}{*}\PY{l+m+mi}{2} \PY{o}{+} \PY{l+m+mi}{1}

        \PY{n}{odd\PYZus{}3x7\PYZus{}spl} \PY{o}{=} \PY{p}{[}\PY{n}{square\PYZus{}plus1}\PY{p}{(}\PY{n}{x}\PY{p}{)} \PY{k}{for} \PY{n}{x} \PY{o+ow}{in} \PY{n}{odd\PYZus{}3x7}\PY{p}{]}
        \PY{c+c1}{\PYZsh{} 리스트의 처음 다섯 개 항목}
        \PY{n}{odd\PYZus{}3x7\PYZus{}spl}\PY{p}{[}\PY{p}{:}\PY{l+m+mi}{5}\PY{p}{]}
\end{Verbatim}

\begin{Verbatim}[commandchars=\\\{\}]
{\color{outcolor}Out[{\color{outcolor}5}]:} [50, 442, 1226, 2402, 3970]
\end{Verbatim}
\fi

\section*{csv 파일 읽어들이기} \label{csv-uxd30cuxc77c-uxc77duxc5b4uxb4e4uxc774uxae30}

\texttt{\textquotesingle{}Seoul\_pop2.csv\textquotesingle{}} 파일에는
아래 내용이 저장되어 있다.

\begin{Verbatim}
  ### 1949년부터 2010년 사이의 서울과 수도권 인구 증가율(%)
  # 구간,서울,수도권

  1949-1955,9.12,-5.83
  1955-1960,55.88,32.22
  1960-1966,55.12,32.76
  1966-1970,45.66,28.76
  1970-1975,24.51,22.93
  1975-1980,21.38,21.69
  1980-1985,15.27,18.99
  1985-1990,10.15,17.53
  1990-1995,-3.64,8.54
  1995-2000,-3.55,5.45
  2000-2005,-0.93,6.41
  2005-2010,-1.34,3.71
\end{Verbatim}

확장자가 csv인 파일은 데이터를 저장하기 위해 주로 사용한다. csv는
Comma-Separated Values의 줄임말로 데이터가 쉼표(콤마)로 구분되어
정리되어 있는 파일을 의미한다.
csv 파일을 읽어드리는 방법은 \texttt{csv} 모듈의 \texttt{reader()}
함수를 활용하면 매우 쉽다. \texttt{reader()} 함수의 리턴값은 csv 파일에
저장된 내용을 줄 단위로, 쉼표 단위로 끊어서 2차원 리스트이다.
예를 들어, 아래 코드는 언급된 파일에 저장된 내용의 각 줄을 출력해준다.

    \begin{Verbatim}[commandchars=\\\{\}]
\PY{k+kn}{import} \PY{n+nn}{csv}
\PY{k}{with} \PY{n+nb}{open}\PY{p}{(}\PY{l+s+s1}{\PYZsq{}}\PY{l+s+s1}{Seoul\PYZus{}pop2.csv}\PY{l+s+s1}{\PYZsq{}}\PY{p}{,} \PY{l+s+s1}{\PYZsq{}}\PY{l+s+s1}{rb}\PY{l+s+s1}{\PYZsq{}}\PY{p}{)} \PY{k}{as} \PY{n}{f}\PY{p}{:}
    \PY{n}{reader} \PY{o}{=} \PY{n}{csv}\PY{o}{.}\PY{n}{reader}\PY{p}{(}\PY{n}{f}\PY{p}{)}
    \PY{k}{for} \PY{n}{row} \PY{o+ow}{in} \PY{n}{reader}\PY{p}{:}
        \PY{k}{if} \PY{n+nb}{len}\PY{p}{(}\PY{n}{row}\PY{p}{)} \PY{o}{==} \PY{l+m+mi}{0} \PY{o+ow}{or} \PY{n}{row}\PY{p}{[}\PY{l+m+mi}{0}\PY{p}{]}\PY{p}{[}\PY{l+m+mi}{0}\PY{p}{]} \PY{o}{==} \PY{l+s+s1}{\PYZsq{}}\PY{l+s+s1}{\PYZsh{}}\PY{l+s+s1}{\PYZsq{}}\PY{p}{:}
            \PY{k}{continue}
        \PY{k}{else}\PY{p}{:}
            \PY{k}{print}\PY{p}{(}\PY{n}{row}\PY{p}{)}
\end{Verbatim}

\subsection*{문제 7}\label{uxbb38uxc81c}

위 코드에서 5번 째 줄을 아래와 같이 하면 오류 발생한다.

\begin{verbatim}
if row[0][0] == '#' or len(row) == 0:
\end{verbatim}

이유를 간단하게 설명하라.

\begin{verbatim}



\end{verbatim}

\section*{넘파이 활용 기초 1} \label{uxb118uxd30cuxc774-uxd65cuxc6a9-uxae30uxcd08-1}

넘파이 어레이를 생성하는 방법은 몇 개의 기본적인 함수를 이용하면 된다.

\begin{itemize}
\tightlist
\item
  \texttt{np.arange()}
\item
  \texttt{np.zeros()}
\item
  \texttt{np.ones()}
\item
  \texttt{np.diag()}
\end{itemize}

\iffalse
예제:

\begin{Verbatim}[commandchars=\\\{\}]
{\color{incolor}In [{\color{incolor}7}]:} \PY{n}{np}\PY{o}{.}\PY{n}{arange}\PY{p}{(}\PY{l+m+mi}{3}\PY{p}{,} \PY{l+m+mi}{10}\PY{p}{,} \PY{l+m+mi}{3}\PY{p}{)}
\end{Verbatim}

\begin{Verbatim}[commandchars=\\\{\}]
{\color{outcolor}Out[{\color{outcolor}7}]:} array([3, 6, 9])
\end{Verbatim}

\begin{Verbatim}[commandchars=\\\{\}]
{\color{incolor}In [{\color{incolor}8}]:} \PY{n}{np}\PY{o}{.}\PY{n}{zeros}\PY{p}{(}\PY{p}{(}\PY{l+m+mi}{2}\PY{p}{,}\PY{l+m+mi}{3}\PY{p}{)}\PY{p}{)}
\end{Verbatim}

\begin{Verbatim}[commandchars=\\\{\}]
{\color{outcolor}Out[{\color{outcolor}8}]:} array([[ 0.,  0.,  0.],
               [ 0.,  0.,  0.]])
\end{Verbatim}

\begin{Verbatim}[commandchars=\\\{\}]
{\color{incolor}In [{\color{incolor}9}]:} \PY{n}{np}\PY{o}{.}\PY{n}{ones}\PY{p}{(}\PY{p}{(}\PY{l+m+mi}{2}\PY{p}{,}\PY{p}{)}\PY{p}{)}
\end{Verbatim}

\begin{Verbatim}[commandchars=\\\{\}]
{\color{outcolor}Out[{\color{outcolor}9}]:} array([ 1.,  1.])
\end{Verbatim}

\begin{Verbatim}[commandchars=\\\{\}]
{\color{incolor}In [{\color{incolor}10}]:} \PY{n}{np}\PY{o}{.}\PY{n}{diag}\PY{p}{(}\PY{p}{[}\PY{l+m+mi}{1}\PY{p}{,} \PY{l+m+mi}{2}\PY{p}{,} \PY{l+m+mi}{3}\PY{p}{,} \PY{l+m+mi}{4}\PY{p}{]}\PY{p}{)}
\end{Verbatim}

\begin{Verbatim}[commandchars=\\\{\}]
{\color{outcolor}Out[{\color{outcolor}10}]:} array([[1, 0, 0, 0],
                [0, 2, 0, 0],
                [0, 0, 3, 0],
                [0, 0, 0, 4]])
\end{Verbatim}

\begin{Verbatim}[commandchars=\\\{\}]
{\color{incolor}In [{\color{incolor}11}]:} \PY{n}{np}\PY{o}{.}\PY{n}{ones}\PY{p}{(}\PY{p}{(}\PY{l+m+mi}{3}\PY{p}{,}\PY{l+m+mi}{3}\PY{p}{)}\PY{p}{)} \PY{o}{*} \PY{l+m+mi}{2}
\end{Verbatim}

\begin{Verbatim}[commandchars=\\\{\}]
{\color{outcolor}Out[{\color{outcolor}11}]:} array([[ 2.,  2.,  2.],
                [ 2.,  2.,  2.],
                [ 2.,  2.,  2.]])
\end{Verbatim}
\fi

\subsection*{문제 8}\label{uxbb38uxc81c}

아래 모양의 어레이를 생성하는 코드를 작성하라. 단, 언급된 네 개의
함수들만 사용해야 하며, 수동으로 생성된 리스트나 어레이는 허용되지
않는다.

\[\left [ \begin{matrix} 2 & 0 & 0 \\ 0 & 2 & 0 \\ 0 & 0 & 2 \end{matrix} \right ]\]

\begin{verbatim}



\end{verbatim}

\iffalse
견본답안:

\begin{Verbatim}[commandchars=\\\{\}]
{\color{incolor}In [{\color{incolor}12}]:} \PY{n}{np}\PY{o}{.}\PY{n}{diag}\PY{p}{(}\PY{n}{np}\PY{o}{.}\PY{n}{ones}\PY{p}{(}\PY{p}{(}\PY{l+m+mi}{3}\PY{p}{,}\PY{p}{)}\PY{p}{)}\PY{o}{*}\PY{l+m+mi}{2}\PY{p}{)}
\end{Verbatim}

            \begin{Verbatim}[commandchars=\\\{\}]
{\color{outcolor}Out[{\color{outcolor}12}]:} array([[ 2.,  0.,  0.],
                [ 0.,  2.,  0.],
                [ 0.,  0.,  2.]])
\end{Verbatim}
\fi

\iffalse
\subsection*{문제}\label{uxbb38uxc81c}

아래 모양의 어레이를 생성하는 코드를 작성하라. 단, 언급된 네 개의 함수만
사용해야 하며, 수동으로 생성된 리스트나 어레이는 허용되지 않는다.

\[\left [ \begin{matrix} 2 & 0 & 0 \\ 0 & 4 & 0 \\ 0 & 0 & 6 \end{matrix} \right ]\]

\begin{verbatim}




\end{verbatim}

견본답안:

    \begin{Verbatim}[commandchars=\\\{\}]
{\color{incolor}In [{\color{incolor}13}]:} \PY{n}{np}\PY{o}{.}\PY{n}{diag}\PY{p}{(}\PY{n}{np}\PY{o}{.}\PY{n}{arange}\PY{p}{(}\PY{l+m+mi}{2}\PY{p}{,} \PY{l+m+mi}{7}\PY{p}{,} \PY{l+m+mi}{2}\PY{p}{)}\PY{p}{)}
\end{Verbatim}

            \begin{Verbatim}[commandchars=\\\{\}]
{\color{outcolor}Out[{\color{outcolor}13}]:} array([[2, 0, 0],
                [0, 4, 0],
                [0, 0, 6]])
\end{Verbatim}

\section*{\texorpdfstring{넘파이의 \texttt{linspace()} 함수
활용}{넘파이의 linspace() 함수 활용}}\label{uxb118uxd30cuxc774uxc758-linspace-uxd568uxc218-uxd65cuxc6a9}

    numpy 모듈의 \texttt{linspace()} 함수는 지정된 구간을 정해진 크기로
일정하게 쪼개는 어래이를 생성한다. 예를 들어, 0부터 3사이의 구간을
균등하게 30개로 쪼개고자 하면 아래와 같이 실행하면 된다.

    \begin{Verbatim}[commandchars=\\\{\}]
{\color{incolor}In [{\color{incolor}14}]:} \PY{n}{xs} \PY{o}{=} \PY{n}{np}\PY{o}{.}\PY{n}{linspace}\PY{p}{(}\PY{l+m+mi}{0}\PY{p}{,} \PY{l+m+mi}{3}\PY{p}{,} \PY{l+m+mi}{30}\PY{p}{)}
         \PY{n}{xs}
\end{Verbatim}

            \begin{Verbatim}[commandchars=\\\{\}]
{\color{outcolor}Out[{\color{outcolor}14}]:} array([ 0.        ,  0.10344828,  0.20689655,  0.31034483,  0.4137931 ,
                 0.51724138,  0.62068966,  0.72413793,  0.82758621,  0.93103448,
                 1.03448276,  1.13793103,  1.24137931,  1.34482759,  1.44827586,
                 1.55172414,  1.65517241,  1.75862069,  1.86206897,  1.96551724,
                 2.06896552,  2.17241379,  2.27586207,  2.37931034,  2.48275862,
                 2.5862069 ,  2.68965517,  2.79310345,  2.89655172,  3.        ])
\end{Verbatim}

    \subsection*{문제}\label{uxbb38uxc81c}

0부터 1사이의 구간을 균등하게 10개로 쪼개어 각 항목을 제곱하는 코드를
작성하라.

    \begin{verbatim}



.
\end{verbatim}

    견본답안:

    \begin{Verbatim}[commandchars=\\\{\}]
{\color{incolor}In [{\color{incolor}15}]:} \PY{n}{np}\PY{o}{.}\PY{n}{linspace}\PY{p}{(}\PY{l+m+mi}{0}\PY{p}{,}\PY{l+m+mi}{1}\PY{p}{,} \PY{l+m+mi}{10}\PY{p}{)} \PY{o}{*}\PY{o}{*} \PY{l+m+mi}{2}
\end{Verbatim}

            \begin{Verbatim}[commandchars=\\\{\}]
{\color{outcolor}Out[{\color{outcolor}15}]:} array([ 0.        ,  0.01234568,  0.04938272,  0.11111111,  0.19753086,
                 0.30864198,  0.44444444,  0.60493827,  0.79012346,  1.        ])
\end{Verbatim}
\fi

\section*{넘파이 활용 기초
2}\label{uxb118uxd30cuxc774-uxd65cuxc6a9-uxae30uxcd08-2}

    \texttt{population.txt} 파일은 1900년부터 1920년까지 캐나다 북부지역에서
서식한 산토끼(hare)와 스라소니(lynx)의 숫자, 그리고 채소인
당근(carrot)의 재배숫자를 아래 내용으로 순수 텍스트 데이터로 담고 있다.

\begin{verbatim}
# year  hare    lynx     carrot
1900    30e3    4e3      48300
1901    47.2e3  6.1e3    48200
1902    70.2e3  9.8e3    41500
1903    77.4e3  35.2e3   38200
1904    36.3e3  59.4e3   40600
1905    20.6e3  41.7e3   39800
1906    18.1e3  19e3     38600
1907    21.4e3  13e3     42300
1908    22e3    8.3e3    44500
1909    25.4e3  9.1e3    42100
1910    27.1e3  7.4e3    46000
1911    40.3e3  8e3      46800
1912    57e3    12.3e3   43800
1913    76.6e3  19.5e3   40900
1914    52.3e3  45.7e3   39400
1915    19.5e3  51.1e3   39000
1916    11.2e3  29.7e3   36700
1917    7.6e3   15.8e3   41800
1918    14.6e3  9.7e3    43300
1919    16.2e3  10.1e3   41300
1920    24.7e3  8.6e3    47300
\end{verbatim}


    아래 코드는 연도, 토끼 개체수, 스라소리 개체수, 당근 개체수를 따로따로
떼어 내어 각각 어레이로 변환하여 \texttt{year}, \texttt{hares},
\texttt{lynxes}, \texttt{carrots} 변수에 저장하는 코드이다.

    \begin{Verbatim}[commandchars=\\\{\}]
\PY{n}{data} \PY{o}{=} \PY{n}{np}\PY{o}{.}\PY{n}{loadtxt}\PY{p}{(}\PY{l+s+s1}{\PYZsq{}}\PY{l+s+s1}{populations.txt}\PY{l+s+s1}{\PYZsq{}}\PY{p}{)}
\PY{n}{year}\PY{p}{,} \PY{n}{hares}\PY{p}{,} \PY{n}{lynxes}\PY{p}{,} \PY{n}{carrots} \PY{o}{=} \PY{n}{data}\PY{o}{.}\PY{n}{T}
\end{Verbatim}

\subsection*{문제 9}\label{uxbb38uxc81c}

위 코드에서 \texttt{data.T}에 대해 간단하게 설명하라.

\begin{verbatim}



\end{verbatim}

    아래 코드는 토끼, 스라소니, 당근 각각의 개체수의 연도별 변화를
선그래프로 보여주도록 하는 코드이다.

    \begin{Verbatim}[commandchars=\\\{\}]
\PY{n}{plt}\PY{o}{.}\PY{n}{axes}\PY{p}{(}\PY{p}{[}\PY{l+m+mf}{0.2}\PY{p}{,} \PY{l+m+mf}{0.1}\PY{p}{,} \PY{l+m+mf}{0.5}\PY{p}{,} \PY{l+m+mf}{0.8}\PY{p}{]}\PY{p}{)}
\PY{n}{plt}\PY{o}{.}\PY{n}{plot}\PY{p}{(}\PY{n}{year}\PY{p}{,} \PY{n}{hares}\PY{p}{,} \PY{n}{year}\PY{p}{,} \PY{n}{lynxes}\PY{p}{,} \PY{n}{year}\PY{p}{,} \PY{n}{carrots}\PY{p}{)}
\PY{n}{plt}\PY{o}{.}\PY{n}{legend}\PY{p}{(}\PY{p}{(}\PY{l+s+s1}{\PYZsq{}}\PY{l+s+s1}{Hare}\PY{l+s+s1}{\PYZsq{}}\PY{p}{,} \PY{l+s+s1}{\PYZsq{}}\PY{l+s+s1}{Lynx}\PY{l+s+s1}{\PYZsq{}}\PY{p}{,} \PY{l+s+s1}{\PYZsq{}}\PY{l+s+s1}{Carrot}\PY{l+s+s1}{\PYZsq{}}\PY{p}{)}\PY{p}{,} \PY{n}{loc}\PY{o}{=}\PY{p}{(}\PY{l+m+mf}{1.05}\PY{p}{,} \PY{l+m+mf}{0.5}\PY{p}{)}\PY{p}{)}
\end{Verbatim}

\begin{center}
\adjustimage{max size={0.4\linewidth}{0.4\paperheight}}{output_73_1.png}
\end{center}

\iffalse
\subsection*{문제}\label{uxbb38uxc81c}

위 코드에서 사용된
\texttt{plt.plot(year,\ hares,\ year,\ lynxes,\ year,\ carrots)} 를
간단하게 설명하라.

\begin{verbatim}


\end{verbatim}
\fi

\subsection*{문제 10}\label{uxbb38uxc81c}

산토끼, 스라소니, 당근의 예제에서 1900년부터 1921년 사이에 개체별
개체수의 변화에 대해 어떤 분석을 할 수 있는지 그래프를 이용하여 간단하게
설명하라.

\begin{verbatim}




\end{verbatim}


    % Add a bibliography block to the postdoc


    \end{document}
